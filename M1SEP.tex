% Options for packages loaded elsewhere
\PassOptionsToPackage{unicode}{hyperref}
\PassOptionsToPackage{hyphens}{url}
\PassOptionsToPackage{dvipsnames,svgnames,x11names}{xcolor}
%
\documentclass[
  letterpaper,
  DIV=11,
  numbers=noendperiod]{scrartcl}

\usepackage{amsmath,amssymb}
\usepackage{iftex}
\ifPDFTeX
  \usepackage[T1]{fontenc}
  \usepackage[utf8]{inputenc}
  \usepackage{textcomp} % provide euro and other symbols
\else % if luatex or xetex
  \usepackage{unicode-math}
  \defaultfontfeatures{Scale=MatchLowercase}
  \defaultfontfeatures[\rmfamily]{Ligatures=TeX,Scale=1}
\fi
\usepackage{lmodern}
\ifPDFTeX\else  
    % xetex/luatex font selection
\fi
% Use upquote if available, for straight quotes in verbatim environments
\IfFileExists{upquote.sty}{\usepackage{upquote}}{}
\IfFileExists{microtype.sty}{% use microtype if available
  \usepackage[]{microtype}
  \UseMicrotypeSet[protrusion]{basicmath} % disable protrusion for tt fonts
}{}
\makeatletter
\@ifundefined{KOMAClassName}{% if non-KOMA class
  \IfFileExists{parskip.sty}{%
    \usepackage{parskip}
  }{% else
    \setlength{\parindent}{0pt}
    \setlength{\parskip}{6pt plus 2pt minus 1pt}}
}{% if KOMA class
  \KOMAoptions{parskip=half}}
\makeatother
\usepackage{xcolor}
\ifLuaTeX
  \usepackage{luacolor}
  \usepackage[soul]{lua-ul}
\else
  \usepackage{soul}
  
\fi
\setlength{\emergencystretch}{3em} % prevent overfull lines
\setcounter{secnumdepth}{-\maxdimen} % remove section numbering
% Make \paragraph and \subparagraph free-standing
\makeatletter
\ifx\paragraph\undefined\else
  \let\oldparagraph\paragraph
  \renewcommand{\paragraph}{
    \@ifstar
      \xxxParagraphStar
      \xxxParagraphNoStar
  }
  \newcommand{\xxxParagraphStar}[1]{\oldparagraph*{#1}\mbox{}}
  \newcommand{\xxxParagraphNoStar}[1]{\oldparagraph{#1}\mbox{}}
\fi
\ifx\subparagraph\undefined\else
  \let\oldsubparagraph\subparagraph
  \renewcommand{\subparagraph}{
    \@ifstar
      \xxxSubParagraphStar
      \xxxSubParagraphNoStar
  }
  \newcommand{\xxxSubParagraphStar}[1]{\oldsubparagraph*{#1}\mbox{}}
  \newcommand{\xxxSubParagraphNoStar}[1]{\oldsubparagraph{#1}\mbox{}}
\fi
\makeatother


\providecommand{\tightlist}{%
  \setlength{\itemsep}{0pt}\setlength{\parskip}{0pt}}\usepackage{longtable,booktabs,array}
\usepackage{calc} % for calculating minipage widths
% Correct order of tables after \paragraph or \subparagraph
\usepackage{etoolbox}
\makeatletter
\patchcmd\longtable{\par}{\if@noskipsec\mbox{}\fi\par}{}{}
\makeatother
% Allow footnotes in longtable head/foot
\IfFileExists{footnotehyper.sty}{\usepackage{footnotehyper}}{\usepackage{footnote}}
\makesavenoteenv{longtable}
\usepackage{graphicx}
\makeatletter
\newsavebox\pandoc@box
\newcommand*\pandocbounded[1]{% scales image to fit in text height/width
  \sbox\pandoc@box{#1}%
  \Gscale@div\@tempa{\textheight}{\dimexpr\ht\pandoc@box+\dp\pandoc@box\relax}%
  \Gscale@div\@tempb{\linewidth}{\wd\pandoc@box}%
  \ifdim\@tempb\p@<\@tempa\p@\let\@tempa\@tempb\fi% select the smaller of both
  \ifdim\@tempa\p@<\p@\scalebox{\@tempa}{\usebox\pandoc@box}%
  \else\usebox{\pandoc@box}%
  \fi%
}
% Set default figure placement to htbp
\def\fps@figure{htbp}
\makeatother

\usepackage{amsmath}

\usepackage{amssymb}

\usepackage{amsthm}

\usepackage{bm} % pour les symboles en gras

\usepackage{mathtools} % pour des fonctionnalités mathématiques étendues

\KOMAoption{captions}{tableheading}
\makeatletter
\@ifpackageloaded{tcolorbox}{}{\usepackage[skins,breakable]{tcolorbox}}
\@ifpackageloaded{fontawesome5}{}{\usepackage{fontawesome5}}
\definecolor{quarto-callout-color}{HTML}{909090}
\definecolor{quarto-callout-note-color}{HTML}{0758E5}
\definecolor{quarto-callout-important-color}{HTML}{CC1914}
\definecolor{quarto-callout-warning-color}{HTML}{EB9113}
\definecolor{quarto-callout-tip-color}{HTML}{00A047}
\definecolor{quarto-callout-caution-color}{HTML}{FC5300}
\definecolor{quarto-callout-color-frame}{HTML}{acacac}
\definecolor{quarto-callout-note-color-frame}{HTML}{4582ec}
\definecolor{quarto-callout-important-color-frame}{HTML}{d9534f}
\definecolor{quarto-callout-warning-color-frame}{HTML}{f0ad4e}
\definecolor{quarto-callout-tip-color-frame}{HTML}{02b875}
\definecolor{quarto-callout-caution-color-frame}{HTML}{fd7e14}
\makeatother
\makeatletter
\@ifpackageloaded{caption}{}{\usepackage{caption}}
\AtBeginDocument{%
\ifdefined\contentsname
  \renewcommand*\contentsname{Table of contents}
\else
  \newcommand\contentsname{Table of contents}
\fi
\ifdefined\listfigurename
  \renewcommand*\listfigurename{List of Figures}
\else
  \newcommand\listfigurename{List of Figures}
\fi
\ifdefined\listtablename
  \renewcommand*\listtablename{List of Tables}
\else
  \newcommand\listtablename{List of Tables}
\fi
\ifdefined\figurename
  \renewcommand*\figurename{Figure}
\else
  \newcommand\figurename{Figure}
\fi
\ifdefined\tablename
  \renewcommand*\tablename{Table}
\else
  \newcommand\tablename{Table}
\fi
}
\@ifpackageloaded{float}{}{\usepackage{float}}
\floatstyle{ruled}
\@ifundefined{c@chapter}{\newfloat{codelisting}{h}{lop}}{\newfloat{codelisting}{h}{lop}[chapter]}
\floatname{codelisting}{Listing}
\newcommand*\listoflistings{\listof{codelisting}{List of Listings}}
\makeatother
\makeatletter
\makeatother
\makeatletter
\@ifpackageloaded{caption}{}{\usepackage{caption}}
\@ifpackageloaded{subcaption}{}{\usepackage{subcaption}}
\makeatother

\usepackage{bookmark}

\IfFileExists{xurl.sty}{\usepackage{xurl}}{} % add URL line breaks if available
\urlstyle{same} % disable monospaced font for URLs
\hypersetup{
  pdftitle={DATA Camp M1},
  colorlinks=true,
  linkcolor={blue},
  filecolor={Maroon},
  citecolor={Blue},
  urlcolor={Blue},
  pdfcreator={LaTeX via pandoc}}


\title{DATA Camp M1}
\author{}
\date{}

\begin{document}
\maketitle


\section{Mathématiques}\label{mathuxe9matiques}

\subsection{Algèbre Linéaire et quelques notions en M1 SEP sur les
matrices}\label{alguxe8bre-linuxe9aire-et-quelques-notions-en-m1-sep-sur-les-matrices}

\subsubsection{Trace d'une matrice}\label{trace-dune-matrice}

La \textbf{trace} d'une matrice carrée \(A\) est la somme de ses
éléments diagonaux : \[\text{Tr}(A) = \sum_{i=1}^{n} a_{ii}\]

Propriétés :

\begin{itemize}
\tightlist
\item
  La trace est linéaire :
  \[\text{Tr}(A + B) = \text{Tr}(A) + \text{Tr}(B)\]
\item
  Invariance par similitude : \[\text{Tr}(AB) = \text{Tr}(BA)\]
\end{itemize}

\subsubsection{Inverse d'une matrice}\label{inverse-dune-matrice}

Une matrice carré \(A\) d'ordre \(n\) est dite inversible s'il existe
\(B\) tel que : \[AB = BA = I_n\]

B est alors noté \(A^{-1}\) l'inverse de A.

\begin{tcolorbox}[enhanced jigsaw, rightrule=.15mm, bottomrule=.15mm, opacitybacktitle=0.6, leftrule=.75mm, colbacktitle=quarto-callout-tip-color!10!white, colback=white, opacityback=0, toprule=.15mm, left=2mm, title=\textcolor{quarto-callout-tip-color}{\faLightbulb}\hspace{0.5em}{Tip}, breakable, bottomtitle=1mm, colframe=quarto-callout-tip-color-frame, toptitle=1mm, titlerule=0mm, coltitle=black, arc=.35mm]

Si le déterminant d'une matrice \(A\) est différent de 0 alors la
matrice est inversible.

\end{tcolorbox}

\subsubsection{\texorpdfstring{Inverse d'une matrice
\((2 \times 2)\)}{Inverse d'une matrice (2 \textbackslash times 2)}}\label{inverse-dune-matrice-2-times-2}

Inverse d'une matrice \(A\in \mathcal{M}_{2 \times 2}\) de déterminant
non nul (\(det(A) = ad -bc \ne 0\)) :

\[A^{-1} = \left(\begin{array}{cc} a & b \\ c & d \end{array} \right)^{-1}
= \frac{1}{ad-bc}\left(\begin{array}{cc} d & -b \\ a & -c \end{array} \right)
\]

\subsubsection{Inverse d'une matrice
diagonale}\label{inverse-dune-matrice-diagonale}

Inverse d'une matrice \(A\in \mathcal{M}_{n \times n}\) de déterminant
non nul (\(det(A) = d_1 \times d_2 \times \cdots \times d_n \ne 0\)) :
\[D^{-1} = 
\left(\begin{array}{cccc}
d_1 & 0 & \cdots & 0 \\
0 & d_2 & \cdots & 0 \\
\vdots & \vdots & \ddots & \vdots \\
0 & 0 & \cdots & d_n
\end{array}  \right)^{-1}
= \left(\begin{array}{cccc}
\frac{1}{d_1} & 0 & \cdots & 0 \\
0 & \frac{1}{d_2} & \cdots & 0 \\
\vdots & \vdots & \ddots & \vdots \\
0 & 0 & \cdots & \frac{1}{d_n}
\end{array} \right)\]

\subsubsection{Espace vectoriel}\label{espace-vectoriel}

Un espace vectoriel est un ensemble de vecteurs qui peuvent être
combinés entre eux par des opérations comme l'addition et la
multiplication. Ces combinaisons permettent de construire de nouveaux
vecteurs dans le même ensemble. Un vecteur \(v = (v_1,...,v_d)\) n'a pas
de dimension mais admet une longueur de taille \(d\) ( = d
coefficients). Mais, il est d'usage de représenter \(v\) avec sa
représentation matricielle notée \(V = \left(\begin{array}{c} v_1 \\
\cdots \\
v_d
\end{array}
\right)\), matrice de \(d\) lignes et une colonne. On écrit donc
\(V = v^t\).

Soient \(v,u\) deux vecteurs d'un espace vectoriel réel de dimension
finie muni d'un \textbf{produit scalaire} usuel :
\[\langle v, u \rangle_2 = v_1 u_1 + v_2 u_2 + \cdots + v_n u_n\] (le
produit scalaire est associée à la norme 2
\(\langle v, v \rangle_2 = \|v\|_2^2\))

Quelques propriétés importantes du produit scalaire :

\begin{itemize}
\item
  \(\langle v, v \rangle_2 = \|v\|_2^2 = \sum_{i=1}^{n} v_i^2\)
\item
  \(\langle v, u \rangle_2 = \langle u, v \rangle_2\) (symétrie)
\item
  \(\langle v, u \rangle_2 > 0\) si \(x\) \ne \(0\)
\item
  \(\langle v, w \rangle_2 = v^t w = \text{trace}(vw^t)\) -
  \(\langle v, w \rangle_2 = \langle w, v \rangle_2\)
\end{itemize}

Quelques inégalités à connaître :

\begin{itemize}
\tightlist
\item
  Inégalité de Cauchy-Schwarz :
  \[|\langle x, y \rangle| \leqslant \|x\| \|y\|\]
\item
  Inégalité triangulaire : \[\|x + y\| \leqslant \|x\| + \|y\|\]
\end{itemize}

\subsubsection{Orthogonalité}\label{orthogonalituxe9}

Soient \(v\) et \(w\), deux vecteurs appartenant à un espace vectoriel
\(E\) de dimension \(n\), muni du produit scalaire
\(\langle \cdot, \cdot \rangle\).

\paragraph{Normalisation d'un vecteur}\label{normalisation-dun-vecteur}

La normalisation d'un vecteur \(v\) est donnée par : \[
x = \frac{v}{\|v\|_2}
\] Ainsi, \(\|v\|_2 = 1\).

Deux vecteurs \(v\) et \(u\) sont \textbf{orthogonaux} si leur produit
scalaire est nul : \[\langle v, u \rangle = 0\] et on note \(v \perp u\)

\paragraph{Matrice de projection
orthogonale}\label{matrice-de-projection-orthogonale}

Soit \(E\) un espace vectoriel muni d'un produit scalaire
\(\langle \cdot, \cdot \rangle\) et \(W\) un sous-espace vectoriel de
\(E\). La projection orthogonale d'un élément \(B\) de \(E\) sur \(W\)
est définie par : \[\Pi_W B = \underset{a \in W}{argmin} \|B - a\|_2\]
La matrice de projection orthogonale de \(E\) sur \(W\) est notée
\(\Pi_W\).

\subparagraph{Propriétés}\label{propriuxe9tuxe9s}

\begin{itemize}
\tightlist
\item
  \(\Pi_W^t = \Pi_W\)
\item
  \(\Pi_W^2 = \Pi_W\)
\item
  \(\text{trace}(\Pi_W) = \dim(\Pi_W)\)
\item
  \(\Pi_W^\perp = I_E - \Pi_W\)
\end{itemize}

Pour tout \(B\), élément de \(E\) : \[\Pi_{W^\perp} B = B - \Pi_{W} B\]

\paragraph{Construction de matrice de projection
orthogonale}\label{construction-de-matrice-de-projection-orthogonale}

Une base \(E\) est un ensemble de vecteurs linéairement indépendants qui
permet de décrire tous les vecteurs d'un espace (par combinaison
linéaire). On note alors
\(E = \text{Vect}(\mathbf{v_1}, \mathbf{v_2}, \dots, \mathbf{v_n})\).
Cette notation désigne l'ensemble de tous les vecteurs qui peuvent
s'écrire comme une combinaison linéaire de
\(\mathbf{v_1}, \mathbf{v_2}, \dots, \mathbf{v_n}\).

Soient \(v_1, \ldots, v_p\), \(p\) vecteurs de \(\mathbb{R}^n\) avec
\(p \leqslant n\). Si \((v_1, \ldots, v_p)\) est une base orthogonale de
\(\mathbb{R}^p\) et \(\text{W} = \text{vect}(v_1, \ldots, v_p)\), alors
la matrice de projection W de \(\mathbb{R}^n\) sur W est :
\[\text{W} = V (V^t V)^{-1} V^t
\] où \(V\) est la matrice dont les colonnes sont les vecteurs
\(v_1, \ldots, v_p\).

Si \((v_1, \ldots, v_p)\) est une base orthonormale de W, alors : \[
\text{W} = V V^t
\]

\paragraph{Procédé d'orthogonalisation de
Gram-Schmidt}\label{procuxe9duxe9-dorthogonalisation-de-gram-schmidt}

Le but du procédé de Gram-Schmidt est de prendre un ensemble de vecteurs
linéairement indépendants \(( {v_1, v_2, \ldots, v_n})\) et de produire
un ensemble de vecteurs orthogonaux \(( {u_1, u_2, \ldots, u_n} )\) qui
engendrent le même sous-espace.

\textbf{Étapes du Procédé}

Le premier vecteur orthogonal ( \(u_1\) ) est le premier vecteur de
l'ensemble original : \[u_1 = v_1\]

Pour chaque vecteur \(( v_k )\) (où \$ k \geq 2 \$), on soustrait les
projections orthogonales de \(( v_k )\) sur les vecteurs orthogonaux
précédemment calculés \(( u_1, u_2, \ldots, u_{k-1})\) :
\[ u_k = v_k - \sum_{j=1}^{k-1} \frac{\langle v_k, u_j \rangle}{\langle u_j, u_j \rangle} u_j \]

Si l'on souhaite obtenir une base orthonormale, chaque vecteur
\(( u_k )\) est normalisé pour obtenir \(( e_k )\) :
\[ e_k = \frac{u_k}{||u_k||} \]

\subsubsection{Base duale}\label{base-duale}

Pour une base \(\{e_1, e_2, \ldots, e_n\}\) d'un espace vectoriel \(E\),
la \textbf{base duale} \((e^1, e^2, \ldots, e^n)\) est définie par :
\[e^i(e_j) = \delta_{ij}\] où \(\delta_{ij}\) est le symbole de
Kronecker.

La base duale permet de définir des formes linéaires et de travailler
avec des espaces vectoriels de manière plus abstraite.

Pour plus d'informations sur la base duale cliquez
\href{https://www.math.univ-paris13.fr/~schwartz/L2/dual.pdf}{\textbf{ici}}

\subsubsection{Diagonalisation d'une
matrice}\label{diagonalisation-dune-matrice}

Une matrice carrée \(A\) est dite \textbf{diagonalisable} s'il existe
une matrice diagonale \(D\) et une matrice inversible \(P\) telles que :
\[A = PDP^{-1}\] \(D\) est une matrice diagonale contenant les valeurs
propres de \(A\) et \(P\) est formé de vecteurs propres dans l'ordre des
valeurs propres mis dans \(D\).

\begin{tcolorbox}[enhanced jigsaw, rightrule=.15mm, bottomrule=.15mm, opacitybacktitle=0.6, leftrule=.75mm, colbacktitle=quarto-callout-tip-color!10!white, colback=white, opacityback=0, toprule=.15mm, left=2mm, title=\textcolor{quarto-callout-tip-color}{\faLightbulb}\hspace{0.5em}{Tip}, breakable, bottomtitle=1mm, colframe=quarto-callout-tip-color-frame, toptitle=1mm, titlerule=0mm, coltitle=black, arc=.35mm]

Si les vecteurs propres sont normalisés alors \(P^{-1} = P^t\) et
\(A = PDP^t\).

\end{tcolorbox}

\paragraph{Diagonalisation d'une matrice symétrique
réelle}\label{diagonalisation-dune-matrice-symuxe9trique-ruxe9elle}

Pour les matrices symétriques réelles, on peut toujours trouver une base
orthonormale de vecteurs propres, ce qui permet de les diagonaliser par
une matrice orthogonale \(Q\) : \[A = QDQ^T\]

\subsection{Probabilités}\label{probabilituxe9s}

\subsubsection{Qu'est-ce qui caractérise une variable aléatoire
?}\label{quest-ce-qui-caractuxe9rise-une-variable-aluxe9atoire}

En théorie, on représente la moyenne comme l'espérance : Dans le cas
discret : \[E(X) = \sum k \cdot P(X = k)\] \[k \in X(\Omega)\] Alors que
dans le cas continu : \[E(X) = \int x \cdot f_X(x) \, dx\]
\[x \in X(\Omega)\]

\paragraph{Fonction de répartition}\label{fonction-de-ruxe9partition}

La fonction de répartition est définie par :
\[F_X(x) = P(X \leqslant x)\]

Pour tout \(x\), \(0 \leqslant F_X(x) \leqslant 1\), \(F_X\) est une
fonction croissante. De plus, \(\lim_{x \to -\infty} F_X(x) = 0\) et
\(\lim_{x \to \infty} F_X(x) = 1\)

\begin{center}
\includegraphics[width=0.5\linewidth,height=\textheight,keepaspectratio]{images/fonction_repartition.png}
\end{center}

\paragraph{Loi marginale}\label{loi-marginale}

La loi marginale de X est définie comme suit :
\[f_X(x) = \int f_{X, Y}(x, y) \, dy, \text{ où } -\infty < x < \infty,\]
dans le cas continu, ou encore :
\[f_X(x_i) = \sum p_{ij}, \text{ où } j \text{ tel que } y_j \leqslant y\]

Si X et Y sont indépendants, alors :
\[f_{X, Y}(x, y) = f_X(x) \cdot f_Y(y)\] \newpage

\subsubsection{Savoir-faire de
probabilités}\label{savoir-faire-de-probabilituxe9s}

\paragraph{Centrage et réduction}\label{centrage-et-ruxe9duction}

Le centrage consiste à localiser la distribution autour de l'origine et
la réduction consiste à normaliser la dispersion. On crée une nouvelle
variable aléatoire \(Y\) issu de \(X\) dont l'espérance est null et la
variable est égale à 1. \[Y = \frac{X - E(X)}{\sqrt{\sigma(X)^2}}\] où
\(E(X)\) représente l'espérance de X et \(\sigma^2\) est la variance de
\(X\).

\paragraph{Moments d'ordre r}\label{moments-dordre-r}

Le moment d'ordre r est défini par : \[\mu_r = E(X^r)\] Le moment centré
d'ordre r est défini par : \[\mũ_r = E((X - E(X))^r)\]

\paragraph{Couples aléatoires}\label{couples-aluxe9atoires}

\subparagraph{Fonction de répartition et
densité}\label{fonction-de-ruxe9partition-et-densituxe9}

La fonction conjointe est appelée la distribution conjointe de X et Y.
\[F_{X, Y}(x, y) = P(X \leqslant x \cap Y \leqslant y)\]

Dans le cas continu, la fonction définie par :
\[f_{X, Y}(x, y) = \frac{\partial^2 F_{X, Y}(x, y)}{\partial x \partial y}\]
est la densité conjointe du couple (X, Y). On a donc :
\[F_{X, Y}(x, y) = \int \int f_{X, Y}(t, u) \, dt \, du, \text{ où } -\infty < x, y < +\infty,\]

Dans le cas discret, on définit la fonction de probabilité conjointe :
\[P(X = x_i, Y = y_j) = p_{ij}\] On a donc :
\[F_{X, Y}(x, y) = \sum \sum p_{ij}, \text{ où } x_i \leqslant x \text{ et } y_j \leqslant y\]

\subparagraph{Covariance}\label{covariance}

La covariance mesure l'intensité de la relation linéaire entre deux
variables aléatoires X et Y. Elle est définie comme suit :
\[Cov(X, Y) = E(XY) - E(X) \cdot E(Y)\]

Si X et Y sont indépendants, alors : \[Cov(X, Y) = 0\]

\begin{tcolorbox}[enhanced jigsaw, rightrule=.15mm, bottomrule=.15mm, opacitybacktitle=0.6, leftrule=.75mm, colbacktitle=quarto-callout-warning-color!10!white, colback=white, opacityback=0, toprule=.15mm, left=2mm, title=\textcolor{quarto-callout-warning-color}{\faExclamationTriangle}\hspace{0.5em}{Warning}, breakable, bottomtitle=1mm, colframe=quarto-callout-warning-color-frame, toptitle=1mm, titlerule=0mm, coltitle=black, arc=.35mm]

Il est important de noter que la réciproque n'est pas vraie : la
covariance n'implique pas nécessairement l'indépendance entre X et Y.

\end{tcolorbox}

\newpage

\subsubsection{Propriétés à connaître de l'espérance, variance et
covariance}\label{propriuxe9tuxe9s-uxe0-connauxeetre-de-lespuxe9rance-variance-et-covariance}

\textbf{Espérance} \[
\mathbb{E}(aX + bY) = a\mathbb{E}(X) + b\mathbb{E}(Y)
\]

\[
\mathbb{E}(a) = a
\] \textbf{Variance}

\[
\text{Var}(aX) = a^2\text{Var}(X)
\] \[
\text{Var}(a) = 0
\] \[
\text{Var}(X + Y) = \text{Var}(X) + \text{Var}(Y) + 2\text{Cov}(X,Y)
\]

\[ \text{Si } X \coprod Y \text{ alors } \text{Var}(X+Y) = \text{Var}(X) + \text{Var}(Y)\]

\textbf{Covariance} \[
\text{Cov}(X, Y) = \text{Cov}(Y, X)
\]

\[
\text{Cov}(aX + bY, cU + dV) = ac\text{Cov}(X, U) + ad\text{Cov}(X, V) + bc\text{Cov}(Y, U) + bd\text{Cov}(Y, V)
\]

\newpage

\subsubsection{Vecteurs aléatoires}\label{vecteurs-aluxe9atoires}

L'espérance d'un vecteur aléatoire \(X = (X_1, X_2, \vdots, X_n)^t\) de
X est le vecteur des espérances de \(X_i\) c'est-à-dire
\(\mathbb{E}(X) = \left(\begin{array}{c} E(X_1) \\ \ldots \\ \mathbb{E}(X_n) \end{array} \right)\).

L'espérance reste linéaire, en particulier pour \(X\) et \(Y\)
appartenant à \(R^n\) on a :

\[\mathbb{E}(aX + bY) = a\mathbb{E}(X) + b\mathbb{E}(Y) \text{ où } a,b \in \mathbb{R}\]
\[\mathbb{E}(AX + B) = A\mathbb{E}(X)+B\]

La variance-covariance de X notée \(\Sigma\) une matrice \(n \times n\)
où chaque élément \(\Sigma_{ij}\) représente la covariance entre les
variables aléatoires \(X_i\) et \(X_j\). Les éléments diagonaux de
\(\Sigma\) correspondent aux variances des composantes \(X_i\), tandis
que les éléments hors-diagonaux représentent les covariances entre les
différentes composantes de \(X\). La matrice \(\Sigma\) est donc définie
comme suit :

\[
\Sigma = \begin{pmatrix}
\text{Var}(X_1) & \text{Cov}(X_1, X_2) & \dots & \text{Cov}(X_1, X_n) \\
\text{Cov}(X_2, X_1) & \text{Var}(X_2) & \dots & \text{Cov}(X_2, X_n) \\
\vdots & \vdots & \ddots & \vdots \\
\text{Cov}(X_n, X_1) & \text{Cov}(X_n, X_2) & \dots & \text{Var}(X_n)
\end{pmatrix}
\] \[ \Sigma_{AX + B} = A\Sigma_X(X)A^t\] où \(\Sigma\) est la variance
de X et A,B sont des matrices (deterministe)

\subsubsection{Lois usuelles}\label{lois-usuelles}

\begin{tcolorbox}[enhanced jigsaw, rightrule=.15mm, bottomrule=.15mm, opacitybacktitle=0.6, leftrule=.75mm, colbacktitle=quarto-callout-tip-color!10!white, colback=white, opacityback=0, toprule=.15mm, left=2mm, title=\textcolor{quarto-callout-tip-color}{\faLightbulb}\hspace{0.5em}{Tip}, breakable, bottomtitle=1mm, colframe=quarto-callout-tip-color-frame, toptitle=1mm, titlerule=0mm, coltitle=black, arc=.35mm]

Les variables \(X_1, \ldots, X_n\) sont indépendantes et identiquement
distribuées (i.i.d.) si et seulement si :

Dans le cas discret :
\[P(X_1 = x_1, \ldots, X_n = x_n) = P(X_1 = x_1) \times \ldots \times P(X_n = x_n)\]

Dans le cas continu :

\[P(X_1 \leq x_1, X_2 \leq x_2, \ldots, X_n \leq x_n) = P(X_1 \leq x_1) \cdot P(X_2 \leq x_2) \cdots P(X_n \leq x_n)\]

\end{tcolorbox}

Ces tableaux récapitulent les lois usuelles que vous pourrez rencontrer
dans différents cours du master.

\begin{longtable}[]{@{}
  >{\raggedright\arraybackslash}p{(\linewidth - 10\tabcolsep) * \real{0.0889}}
  >{\raggedright\arraybackslash}p{(\linewidth - 10\tabcolsep) * \real{0.2444}}
  >{\raggedright\arraybackslash}p{(\linewidth - 10\tabcolsep) * \real{0.1556}}
  >{\raggedright\arraybackslash}p{(\linewidth - 10\tabcolsep) * \real{0.2148}}
  >{\raggedright\arraybackslash}p{(\linewidth - 10\tabcolsep) * \real{0.1481}}
  >{\raggedright\arraybackslash}p{(\linewidth - 10\tabcolsep) * \real{0.1481}}@{}}
\toprule\noalign{}
\begin{minipage}[b]{\linewidth}\raggedright
Nom
\end{minipage} & \begin{minipage}[b]{\linewidth}\raggedright
Notation
\end{minipage} & \begin{minipage}[b]{\linewidth}\raggedright
\(X(\Omega)\)
\end{minipage} & \begin{minipage}[b]{\linewidth}\raggedright
\(P(X = k)\)
\end{minipage} & \begin{minipage}[b]{\linewidth}\raggedright
\(E[X]\)
\end{minipage} & \begin{minipage}[b]{\linewidth}\raggedright
\(Var(X)\)
\end{minipage} \\
\midrule\noalign{}
\endhead
\bottomrule\noalign{}
\endlastfoot
Uniforme & \(X \sim U(\{1, 2, \ldots, n\})\) & \(\{1, 2, \ldots, n\}\) &
\(\frac{1}{n}\) & \(\frac{n+1}{2}\) & \(\frac{n^2-1}{12}\) \\
Bernouilli & \(X \sim B(p), 0 < p < 1\) & \(\{0, 1\}\) &
\(P(X = 1) = p\) & \(p\) & \(p(1-p)\) \\
& & & \(P(X = 0) = 1 - p\) & & \\
Binomiale & \(X \sim B(n, p), 0 < p < 1\) & \(\{1, 2, \ldots, n\}\) &
\(C_k^n p^k (1 - p)^{n-k}\) & \(np\) & \(np(1-p)\) \\
Géométrique & \(X \sim G(p), 0 < p < 1\) & \(\mathbb{N}^{*}\) &
\(p(1-p)^{k}\) & \(\frac{1-p}{p}\) & \(\frac{1-p}{p^2}\) \\
Poisson & \(X \sim P(\lambda), \lambda > 0\) & \(\mathbb{N}\) &
\(\frac{\lambda^k}{k!}e^{-\lambda}\) & \(\lambda\) & \(\lambda\) \\
\end{longtable}

\section{Tableau des lois de
probabilité}\label{tableau-des-lois-de-probabilituxe9}

\begin{longtable}[]{@{}
  >{\raggedright\arraybackslash}p{(\linewidth - 10\tabcolsep) * \real{0.0850}}
  >{\raggedright\arraybackslash}p{(\linewidth - 10\tabcolsep) * \real{0.1741}}
  >{\raggedright\arraybackslash}p{(\linewidth - 10\tabcolsep) * \real{0.0688}}
  >{\raggedright\arraybackslash}p{(\linewidth - 10\tabcolsep) * \real{0.3644}}
  >{\raggedright\arraybackslash}p{(\linewidth - 10\tabcolsep) * \real{0.1417}}
  >{\raggedright\arraybackslash}p{(\linewidth - 10\tabcolsep) * \real{0.1660}}@{}}
\toprule\noalign{}
\begin{minipage}[b]{\linewidth}\raggedright
Nom
\end{minipage} & \begin{minipage}[b]{\linewidth}\raggedright
Notation
\end{minipage} & \begin{minipage}[b]{\linewidth}\raggedright
\(X(\Omega)\)
\end{minipage} & \begin{minipage}[b]{\linewidth}\raggedright
\(f_X(x)\)
\end{minipage} & \begin{minipage}[b]{\linewidth}\raggedright
\(E[X]\)
\end{minipage} & \begin{minipage}[b]{\linewidth}\raggedright
\(V(X)\)
\end{minipage} \\
\midrule\noalign{}
\endhead
\bottomrule\noalign{}
\endlastfoot
Uniforme & \(X \sim U([a, b]), a < b\) & \([a, b]\) &
\(\frac{1}{b - a}1_{[a, b]}(x)\) & \(\frac{a + b}{2}\) &
\(\frac{(a - b)^2}{12}\) \\
Exponentielle & \(X \sim \mathcal{E}(\lambda), \lambda > 0\) &
\(\mathbb{R}^+\) & \(\lambda e^{-\lambda x}1_{\mathbb{R}^+}(x)\) &
\(\frac{1}{\lambda}\) & \(\frac{1}{\lambda^2}\) \\
& \(\mathcal{E}(\lambda) = \gamma(1, \lambda)\) & & & & \\
Normale ou Gaussienne & \(X \sim N(m, \sigma^2), \sigma > 0\) &
\(\mathbb{R}\) &
\(\frac{1}{\sigma\sqrt{2\pi}} \exp\left(-\frac{(x - m)^2}{2\sigma^2}\right)\)
& \(m\) & \(\sigma^2\) \\
Gamma & \(X \sim \gamma(\alpha, \theta), \alpha > 0, \theta > 0\) &
\(\mathbb{R}^+\) &
\(\frac{\theta^\alpha}{\Gamma(\alpha)}x^{\alpha - 1}e^{-\theta x}1_{\mathbb{R}^+}(x)\)
& \(\frac{\alpha}{\theta}\) & \(\frac{\alpha}{\theta^2}\) \\
& \(\Gamma(\alpha) = \int_0^\infty e^{-x}x^{\alpha-1} \, dx\) & & & & \\
Khi-2 & \(X \sim \chi^2(n), n \in \mathbb{N}^+\) & \(\mathbb{R}^+\) &
\(\gamma\left(\frac{n}{2}, \frac{1}{2}\right)\) & \(n\) & \(2n\) \\
& \(Y_1, Y_2, \ldots, Y_n \text{ i.i.d},\) & & & & \\
& \(Y_i \sim \mathcal{N}(0, 1), \quad X = \sum_{i=1}^{n} Y_i^2\) & & &
& \\
Bêta & \(X \sim B(\alpha, \theta), \alpha > 0, \theta > 0\) & \([0, 1]\)
&
\(\frac{x^{\alpha-1}(1-x)^{\theta-1}}{B(\alpha, \theta)}1_{[0, 1]}(x)\)
& \(\frac{\alpha}{\alpha+\theta}\) &
\(\frac{\alpha\theta}{(\alpha+\theta)^2(\alpha+\theta+1)}\) \\
& \(B(\alpha, \theta) = \int_0^1 x^{\alpha-1}(1-x)^{\theta-1} \, dx\) &
& & & \\
& \(X = \frac{Z}{1 + Z}, \quad Z \sim B'(\alpha, \theta)\) & & & & \\
Bêta (prime) & \(Z ∼ B'(\alpha, \theta), \alpha > 0, \theta > 0\) &
\(\mathbb{R}^+\) &
\(\frac{z^{\alpha-1}}{B(\alpha,\theta) \cdot (1+z)^{\alpha+\theta}} \cdot 1_{\mathbb{R}^+}(z)\)
& \(\frac{\alpha}{\theta - 1}\) &
\(\frac{\alpha(\alpha+\theta-1)}{(\theta-1)^2(\theta-2)}\) \\
&
\(X \sim \gamma(\alpha, 1), Y \sim \gamma(\theta, 1), X \perp\!\!\!\perp Y\)
& & & \(\theta > 1\) & \(\theta > 2\) \\
Student T & \(T \sim T(n), n \in \mathbb{N}^*\) & \(\mathbb{R}\) &
\(\frac{\left(1 + \frac{t^2}{n}\right)^{-(n+\frac{1}{2})}}{\sqrt{n}\cdot B(\frac{1}{2},\frac{n}{2})}\)
& \(0\) & \(\frac{n}{n - 2}\) \\
& \(T = \frac{X}{\sqrt{Y/n}}, \quad T^2/n \sim B'(1/2, n/2)\) & & & &
\(n > 2\) \\
Fisher & \(X \sim F(n, m)\) & \(\mathbb{R}^+\) &
\(\frac{Kx^{n/2-1}}{(m+nx)^{(n+m)/2}}1_{\mathbb{R}^+}(x)\) &
\(\frac{m}{m-2}\) & \(\frac{2m^2(n+m-2)}{n(m-2)^2(m-4)}\) \\
& \(N \sim \chi^2(n), M \sim \chi^2(m), N \perp\!\!\!\perp M\) & & &
\(m > 2\) & \(m > 4\) \\
\end{longtable}

\newpage

Connaître les densités ne servirait pas à grand chose, mais ceci nous
évitera de parler de lois dont nous ne connaissons pas leur
construction.

\newpage

\subsubsection{Notions de convergence}\label{notions-de-convergence}

Si l'on pense à des données, vues comme réalisation de variables
aléatoires \(X_1, \ldots, X_n\), il serait intéressant de se poser la
question de savoir comment évolue cette suite lorsque \(n\) tend vers
l'infini.

On dit que \((X_n)\) converge presque sûrement vers \(X\) et on note
\(X_n \underset{n \rightarrow \infty}{\xrightarrow{\text{p.s.}}}{} X\)
si et seulement si :
\(P\left(\underset{n\to+\infty}{\lim} X_n = X\right) = 1\)

On dit que \((X_n)\) converge en probabilité vers \(X\) et on note
\(X_n \underset{n \rightarrow \infty}{\xrightarrow{\text{p}}} X\) si et
seulement si :
\(\forall \varepsilon > 0, \quad P(|X_n - X| > \varepsilon) \rightarrow 0\)

On dit que \((X_n)\) converge en loi vers \(X\) et on note
\(X_n \xrightarrow{\mathcal{L}} X\) si et seulement si :
\(F_{X_n} \underset{n \to +\infty}{\rightarrow} F_X\) Où \(F_X\) dénote
la fonction de répartition de \(X\).

On dit que \((X_n)\) converge en moyenne quadratique vers \(X\) et on
note \(X_n \underset{n \rightarrow \infty}{\xrightarrow{m.q.}}X\) si et
seulement si : \(\mathbb{E}((X_n - X)^2) \rightarrow 0\)

\begin{center}
\includegraphics[width=0.5\linewidth,height=\textheight,keepaspectratio]{images/convergence.png}
\end{center}

\subsubsection{Loi faible des grands
nombres}\label{loi-faible-des-grands-nombres}

Soit \(X_1, \ldots, X_n\) une suite de variables aléatoires
indépendantes et de même loi telles que : \(\mathbb{E}(X_i) = \mu\) et
\(\text{Var}(X_i) = \sigma^2\) alors :\\
\[\frac{1}{n} \sum_{i=1}^{n} X_i \xrightarrow{P} \mu \]

Soit \(Z_1,\ldots, Z_n\) une suite de vecteurs aléatoires où \(Z_i\) est
un vecteur aléatoire à p dimensions :
\[Z_i = \begin{pmatrix} Z_{i1} \\ Z_{i2} \\ \vdots \\ Z_{ip} \end{pmatrix}\]

On suppose que les vecteurs \(Z_1, Z_2, \dots, Z_n\) sont i.i.d. , avec
une espérance \(\mathbb{E}[Z_i] = \mu\), où \(\mu\) est un vecteur
constant de dimension \(p\) et chaque composante \(Z_{ij}\) a une
variance finie.
\[\mu = \begin{pmatrix} \mu_1 \\ \mu_2 \\ \vdots \\ \mu_p \end{pmatrix}\]

Alors, la \textbf{moyenne empirique} \(\bar{Z}_n\) converge en
\textbf{probabilité} vers l'espérance \(\mu\) lorsque \(n \to \infty\) :

\[\bar{Z}_n  = \frac{1}{n} \sum_{i=1}^{n} Z_i = \begin{pmatrix} \frac{1}{n} \sum_{i=1}^{n} Z_{i1} \\ \frac{1}{n} \sum_{i=1}^{n} Z_{i2} \\ \vdots \\ \frac{1}{n} \sum_{i=1}^{n} Z_{ip} \end{pmatrix} \xrightarrow{P} \mu \quad \text{lorsque} \quad n \to \infty\]

\subsubsection{Théorème central
limite}\label{thuxe9oruxe8me-central-limite}

Soit \(X_1, \ldots, X_n\) une suite de variables aléatoires
indépendantes et de même loi telles que : \(\mathbb{E}(X_i) = \mu\) et
\(\text{Var}(X_i) = \sigma^2\), alors :
\[\sqrt{n}\frac{\overline{X}_n - \mu}{\sigma} \xrightarrow{\mathcal{Loi}} \mathcal{N}(0, 1)\]

Soit \(Z_1,\ldots, Z_n\) une suite de vecteurs aléatoires où \(Z_i\) est
un vecteur aléatoire à p dimensions et si les vecteurs aléatoires
suivent les mêmes hypothèses que précedement alors
\[\Sigma_X^{-\frac{1}{2}}(\bar{Z}_n-\mathbb{E}(Z) )\xrightarrow{\mathcal{Loi}} \mathcal{N}_d(0, 1)\]

\newpage

\subsection{Statistique
inférentielle}\label{statistique-infuxe9rentielle}

\subsubsection{Echantillon / Estimateur}\label{echantillon-estimateur}

Le point de départ est un vecteur (ou un tableau dans le cas
multidimensionnel) de données. Ces données peuvent être vues comme les
réalisations \((x_1, x_2, \ldots, x_n)\) d'une variable aléatoire \(X\)
qui dépend d'un certain paramètre \(\theta\) que nous allons chercher à
estimer. Pour ce faire, nous allons construire un échantillon de cette
variable. Un échantillon \((X_1, X_2, \ldots, X_n)\) est un n-uplet de
variables aléatoires indépendantes qui suivent toutes la même loi (celle
de \(X\)). Un estimateur de \(\theta\) est une fonction
\(\hat{\theta} = f(X_1, X_2, \ldots, X_n)\) de notre échantillon, qui
possède une loi de probabilité. Lorsque l'aléa est réalisé,
\(\hat{\theta}(\omega) = f(x_1, x_2, \ldots, x_n)\) est une estimation
de \(\theta\). Le but de ce cours est de construire le meilleur
estimateur possible de \(\theta\).

\subsubsection{Estimateur sans biais}\label{estimateur-sans-biais}

Pour que l'estimation soit bonne, il faut que \(\hat{\theta}\) soit
proche de \(\theta\). Comme \(\hat{\theta} = f(X_1, X_2, \ldots, X_n)\)
est une variable aléatoire, on ne peut imposer de condition qu'à sa
valeur moyenne.

On définit ainsi le biais :
\[b_n(\hat{\theta}, \theta) = \mathbb{E}(\hat{\theta}_n) - \theta\]

Un estimateur est dit sans biais si \(b_n(\hat{\theta}, \theta) = 0\),
c'est-à-dire : \[\mathbb{E}(\hat{\theta}_n) = \theta\]

Le plus souvent, on veut estimer la moyenne et la variance.

\paragraph{Estimations de la moyenne}\label{estimations-de-la-moyenne}

Soit un échantillon de variables aléatoires \(X_1, X_2, \dots, X_n\)
tirées d'une population avec une moyenne \(\mu = \mathbb{E}[X_i]\).

La moyenne empirique (ou moyenne d'échantillon) est définie par :

\[
\hat{\mu} = \frac{1}{n} \sum_{i=1}^{n} X_i
\]

L'espérance de cet estimateur est :

\[
\mathbb{E}[\hat{\mu}] = \mathbb{E}\left[\frac{1}{n} \sum_{i=1}^{n} X_i\right] = \frac{1}{n} \sum_{i=1}^{n} \mathbb{E}[X_i] = \mu
\]

La moyenne empirique est donc \textbf{non biaisée}. Son espérance est
égale à la vraie moyenne de la population.

\paragraph{Estimation de la variance}\label{estimation-de-la-variance}

Soit \(X_1, X_2, \dots, X_n\) un échantillon de variables aléatoires
\(X_i\) avec une variance \(\sigma^2 = \mathbb{V}[X_i]\).

L'estimateur classique de la variance basé sur l'échantillon est :

\[
\hat{\sigma}^2_{\text{biaisé}} = \frac{1}{n} \sum_{i=1}^{n} (X_i - \hat{\mu})^2
\]

Cependant, cet estimateur est \textbf{biaisé}. En effet, l'espérance de
cet estimateur n'est pas égale à \(\sigma^2\) :

\[
\mathbb{E}[\hat{\sigma}^2_{\text{biaisé}}] = \frac{n-1}{n} \sigma^2
\]

Pour corriger ce biais, on peut utiliser un estimateur non biaisé de la
variance. Cela se fait en divisant la somme des carrés par \(n - 1\) au
lieu de \(n\). Ainsi, l'estimateur non biaisé de la variance est :

\[
\hat{\sigma}^2_{\text{non biaisé}} = \frac{1}{n - 1} \sum_{i=1}^{n} (X_i - \hat{\mu})^2
\]

Cet estimateur est non biaisé, et l'espérance de
\(\hat{\sigma}^2_{\text{non biaisé}}\) est égale à la vraie variance de
la population :

\[
\mathbb{E}[\hat{\sigma}^2_{\text{non biaisé}}] = \sigma^2
\]

\subsubsection{Estimateur convergent}\label{estimateur-convergent}

Un estimateur est dit convergent s'il converge en probabilité vers le
paramètre à estimer : \[\hat{\theta}_n \xrightarrow{P} \theta\]

En pratique, tout estimateur sans biais et dont la variance tend vers 0
est convergent.

\subsubsection{Estimateur optimal}\label{estimateur-optimal}

La qualité d'un estimateur est mesurée à travers son erreur quadratique
moyenne définie par :
\[EQM(\hat{\theta}_n) = (b_n(\hat{\theta}, \theta))^2 + V(\hat{\theta}_n)\]
Comme nous cherchons tout le temps (presque) des estimateurs sans biais,
il reste à comparer les variances.

Un estimateur 𝜃̂1 est meilleur que 𝜃̂2 si :
\[V(\hat{\theta}_1) < V(\hat{\theta}_2)\]

\textcolor{green!70!black}{Inégalité de Rao-Cramer/ Efficacité}

On définit la quantité d'information apportée par l'estimateur par : \[
I(\hat{\theta}_n) = -\left( \mathbb{E} \left( \frac{\partial L}{\partial \theta} \right) \right)^2
\] Où 𝐿(𝑥, 𝜃) = ∏ 𝑓(𝑥𝑖) (nous reviendrons sur sa définition)

L'inégalité de Rao-Cramer postule que la variance d'un estimateur ne
peut pas aller en delà d'un certain seuil :
\[V(\hat{\theta}_n) \geqslant \frac{1}{I(\hat{\theta}_n)}\] Un
estimateur est optimal (ou efficace) si sa variance vérifie le cas
d'égalité.

\newpage

\subsubsection{Construction d'un
estimateur}\label{construction-dun-estimateur}

Un estimateur est une valeur qu'on ne peut pas obtenir en théorie donc
on essaye de l'estimer. Par exemple, on ne peut pas calculer l'espérence
d'une série de données et donc pour essayer d'obtenir une valeur on
calculer une estimation. On recours à plusieurs estimateurs tels que le
maximum de vraisemblance ou bien la méthode des moments.

\begin{tcolorbox}[enhanced jigsaw, rightrule=.15mm, bottomrule=.15mm, opacitybacktitle=0.6, leftrule=.75mm, colbacktitle=quarto-callout-note-color!10!white, colback=white, opacityback=0, toprule=.15mm, left=2mm, title=\textcolor{quarto-callout-note-color}{\faInfo}\hspace{0.5em}{Note}, breakable, bottomtitle=1mm, colframe=quarto-callout-note-color-frame, toptitle=1mm, titlerule=0mm, coltitle=black, arc=.35mm]

La méthode du maximum de vraisemblance est la plus souvent utilisé dans
les modèles de prédiction.

\end{tcolorbox}

La méthode du maximum de vraisemblance consiste à affecter \(𝜃\) la
valeur qui maximise la probabilité d'observer \((𝑥_1, 𝑥_2, … , 𝑥_𝑛)\)
lorsque l'aléa du vecteur \((𝑋_1, 𝑋_2, … , 𝑋_𝑛)\) tombe. Sans trop
rentrer dans la théorie de la vraisemblance, nous allons présenter un
algorithme en cinq étapes pour calculer cet estimateur :

\textbf{Etape 1} : Calculer la fonction de vraisemblance

Dans le cas continu : \[L(\mathbf{x}, \theta) = \prod_{i=1}^{n} f(x_i)\]

Dans le cas discret :
\[L(\mathbf{x}, \theta) = \prod_{i=1}^{n} P(X_i = x_i)\]

\textbf{Etape 2} : Calculer le log-vraisemblance Il s'agit de calculer
un maximum, ce qui revient à dériver. Il s'agit ici d'un produit de n
facteurs, ce qui rend la dérivation assez coriace. La fonction
logarithmique présente des propriétés assez sympas pour faciliter cette
tâche.

\textbf{Etape 3} : Calculer la dérivée de la log-vraisemblance

\textbf{Etape 4} : Résoudre l'équation d'inconnue \(𝜽\)\}
\[\frac{\partial (\ln(L))}{\partial \theta} = 0 \Rightarrow \theta = \theta_0\]

\textbf{Etape 5}: Vérifier qu'il s'agit d'un maximum.

En s'assurant que :
\[\frac{\partial^2 (\ln(L))}{\partial \theta^2} < 0\]

\newpage

La méthode des moments consiste à égaliser les moments théoriques
(espérance, variance) à leurs équivalents empiriques et à en dégager une
estimation ponctuelle.

En pratique, il faut résoudre l'(les) équation(s) :
\[\mathbb{E}(X) = \overline{X} \text{ et } \text{Var}(X) = S_n^2\] avec
: \[\overline{X} = \frac{1}{n} \sum_{i=1}^{n} X_i \hspace{2cm}
 S_n^2 = \frac{1}{n} \sum_{i=1}^{n} (X_i - \overline{X})^2\]

Il existe une autre méthode d'estimateurs qu'est la méthode des moindres
carrés ordinaires.

Lorsqu'il s'agit de prendre une mesure 𝜃 avec un appareil doté d'une
imprécision \(𝜀\), alors le problème d'estimation peut s'écrire :
\(𝑋 = 𝜃 + 𝜀\). La méthode des moindres-carrés ordinaires consiste à
trouver le paramètre \(𝜃\) qui minimise la somme des carrées des erreurs
:
\[𝜃_{𝑀𝐶𝑂} = \arg\min \left( \sum_{i=0}^n \varepsilon_i^2 \right) = \arg\min \left( \sum_{i=0}^n (X_i - \theta)^2 \right)\]

\subsubsection{Intervalles de confiance}\label{intervalles-de-confiance}

Un intervalle de confiance {[}\(A\), \(B\){]} de niveau \(1 - \alpha\)
est un intervalle aléatoire qui a la probabilité \(1 - \alpha\) de
contenir le paramètre à estimer \(\theta\). Formellement, on écrit :
\(P (t_1 (\theta) \leqslant f(X_1, \ldots, X_n) \leqslant t_2 (\theta)) = P(A \leqslant \theta \leqslant B) = 1 - \alpha\)

\newpage

\subsubsection{Test d'hypothèses}\label{test-dhypothuxe8ses}

Dans le cadre d'un test d'hypothèse, nous cherchons à faire valoir une
hypothèse en dépit d'une autre, qui lui est contradictoire.

On appellera la première (celle dont le rejet à tort sera le plus
préjudiciable) « Hypothèse nulle » et la deuxième « Hypothèse
alternative ».

\begin{center}
\includegraphics[width=0.5\linewidth,height=\textheight,keepaspectratio]{images/test_hypo.png}
\end{center}

Les calculs qui se cachent derrière le choix de l'hypothèse à garder
sont compliqués. Mais BONNE NOUVELLE, la machine fera tour à notre
place. Il suffit juste de suivre correctement la méthode :

\textbf{Etape 1} : Choisir judicieusement les hypothèses à évaluer et
fixer le risque \(𝛼\)\\
\textbf{Etape 2} : Choisir le test adapté à la procédure\\
\textbf{Etape 3} : Rentrer la commande correspondante sur R et
exécuter\\
\textbf{Etape 4}: Lire dans les sorties la p-value. si elle est
supérieure à α on \textbf{conserve} l'hypothèse nulle H0. Si elle lui
est inférieure, on \textbf{rejette} H0 et on \textbf{accepte}
l'hypothèse alternative H1.

\newpage

\subsubsection{Construction d'intervalles de
confiance}\label{construction-dintervalles-de-confiance}

Les intervalles de confiance sont des outils essentiels en statistique
pour estimer des paramètres inconnus tout en mesurant l'incertitude
associée à cette estimation. Ci-dessous, vous trouverez un tableau
présentant la construction des intervalles de confiance pour différents
paramètres.

\begin{figure}[H]
  \centering
  \includegraphics[width=0.75\textwidth]{images/intervalles_conf.png}
\end{figure}

\subsection{Statistiques descriptives
univariées}\label{statistiques-descriptives-univariuxe9es}

\begin{itemize}
\item
  La \textbf{moyenne arithmétique} d'une série de valeurs
  \(x_1, x_2, \ldots, x_n\) est donnée par :
  \[\bar{x} = \frac{1}{n} \sum_{i=1}^{n} x_i\]
\item
  La \textbf{médiane} est la valeur qui sépare la série en deux parties
  égales. Pour une série ordonnée, si \(n\) est impair, la médiane est
  la valeur centrale. Si \(n\) est pair, c'est la moyenne des deux
  valeurs centrales.
\item
  La \textbf{variance} est une mesure de la dispersion des valeurs
  autour de la moyenne :
  \[\sigma_x^2 = \frac{1}{n} \sum_{i=1}^{n} (x_i - \bar{x})^2\]
\item
  L'\textbf{écart-type} est la racine carrée de la variance :
  \[\sigma = \sqrt{\sigma_x^2}\]
\end{itemize}

\begin{tcolorbox}[enhanced jigsaw, rightrule=.15mm, bottomrule=.15mm, opacitybacktitle=0.6, leftrule=.75mm, colbacktitle=quarto-callout-note-color!10!white, colback=white, opacityback=0, toprule=.15mm, left=2mm, title=\textcolor{quarto-callout-note-color}{\faInfo}\hspace{0.5em}{Note}, breakable, bottomtitle=1mm, colframe=quarto-callout-note-color-frame, toptitle=1mm, titlerule=0mm, coltitle=black, arc=.35mm]

Plus l'écart-type est grand, plus les données sont dispersées autour de
la moyenne.

\end{tcolorbox}

Les statistiques descriptives univariées sont essentielles pour résumer
et comprendre les caractéristiques principales d'une seule variable.
Elles sont largement utilisées en analyse de données pour obtenir une
vue d'ensemble rapide et efficace. Elles permettent aussi d'identifier
rapidement des valeurs extrêmes.

\subsection{Statistiques descriptives
bivariées}\label{statistiques-descriptives-bivariuxe9es}

\subsubsection{Définitions}\label{duxe9finitions}

\begin{itemize}
\item
  Les nuages de points (ou diagrammes de dispersion) sont une méthode
  graphique utilisée pour observer la relation entre deux variables
  quantitatives. Chaque point du graphique représente une paire de
  valeurs \((x,y)\) pour ces deux variables. Ce type de graphique permet
  de visualiser des tendances, des corrélations (positives, négatives ou
  nulles), ainsi que la présence d'éventuelles anomalies ou valeurs
  aberrantes. Si les points semblent s'aligner le long d'une ligne
  droite, cela peut indiquer une relation linéaire entre les deux
  variables.
\item
  Les boxplots (ou boîtes à moustaches) sont un autre outil graphique,
  souvent utilisé pour représenter la distribution d'une variable
  quantitative en fonction d'une variable qualitative. Ils montrent la
  médiane, les quartiles, ainsi que les valeurs extrêmes ou aberrantes
  d'un jeu de données. Dans un boxplot, la boîte représente l'intervalle
  interquartile (de Q1 à Q3), la ligne médiane à l'intérieur de la boîte
  représente la médiane de la distribution, et les moustaches s'étendent
  jusqu'aux valeurs non aberrantes les plus extrêmes. Ces graphiques
  sont utiles pour comparer rapidement la dispersion et la symétrie des
  distributions entre différentes catégories d'une variable qualitative.
\end{itemize}

\section{Economie}\label{economie}

\subsection{Concepts de base}\label{concepts-de-base}

\subsubsection{Macroéconomie}\label{macrouxe9conomie}

\begin{center}
\includegraphics[width=0.76042in,height=\textheight,keepaspectratio]{images/macroeconomics.png}
\end{center}

La macroéconomie est l'étude économique d'un système ou de phénomènes à
un niveau global de l'économie.

\subsubsection{Microéconomie}\label{microuxe9conomie}

La microéconomie se concentre sur l'observation et l'analyse des
interactions à petite échelle.

\subsubsection{Bien économique}\label{bien-uxe9conomique}

``Chose utile à satisfaire un besoin, il faut que le bien soit
disponible et en quantité limitée.

Un bien non économique est un bien qui s'obtient gratuitement, comme
l'oxygène, contrairement à un bien économique qui s'obtient en payant.''

\subsubsection{Agent économique}\label{agent-uxe9conomique}

``Un agent économique est un individu ou un groupe d'individus
constituant un centre de décision économique indépendant.''

\subsubsection{Marché}\label{marchuxe9}

\begin{center}
\includegraphics[width=0.76042in,height=\textheight,keepaspectratio]{images/economie-circulaire.png}
\end{center}

``Le marché c'est une institution sociale qui permet l'échange entre
l'offre et la demande.''

\subsubsection{Asymétrie d'information}\label{asymuxe9trie-dinformation}

``L'asymétrie d'information concerne les situations où les agents d'un
marché ne possèdent pas de la même information sur un produit que ce
soit au sujet de ses qualités ou de ses défauts''

\subsubsection{Externalité}\label{externalituxe9}

L'externalité désigne la conséquence (positive ou négative) d'une
activité d'un agent économique sur un autre, sans qu'aucun des deux ne
reçoive ou ne paye une compensation pour cet effet.

\subsubsection{Concurrence Pure et
Parfaite}\label{concurrence-pure-et-parfaite}

La \textbf{CPP} repose sur cinq fondements :

\begin{itemize}
\tightlist
\item
  L'Atomicité du marché
\end{itemize}

Existence d'un grand nombre d'agent économique sur le marché, à tel un
point que ni l'offre ni la demande ne peut exercer une action quelconque
sur la production et les prix ;

\begin{itemize}
\tightlist
\item
  L'Homogénéité des produits
\end{itemize}

La préférence d'un produit à un autre du point de vue de l'acheteur se
fait uniquement selon son prix ;

\begin{itemize}
\tightlist
\item
  Libre entrée et sortie sur le marché
\end{itemize}

Aucune firme ne peut s'opposer à l'arrivée d'un concurrent sur le
marché, tout le monde est libre de l'intégrer ;

\begin{itemize}
\tightlist
\item
  Libre circulation des facteurs de production
\end{itemize}

Les facteurs de production (capital et travail) doivent être libre de se
déplacer librement sans obstacle d'une industrie à l'autre ;

\begin{itemize}
\tightlist
\item
  La transparence de l'information
\end{itemize}

Offreurs et demandeurs sont parfaitement conscient des caractéristiques
et prix des produits.

\subsubsection{Monopole}\label{monopole}

``Le monopole est une situation dans un marché où un vendeur fait face
aux multitudes vendeurs.''

\subsubsection{Segmentation de marché}\label{segmentation-de-marchuxe9}

``La segmentation de marché est un découpage du marché en groupes
homogènes selon des critères spécifiques, que ce soit des critères
démographiques ou bien géo-graphiques.''

\subsubsection{Discrimination par les
prix}\label{discrimination-par-les-prix}

``La discrimination par les prix est le pouvoir de pratiquer des prix
différents pour un même produit, peut s'appliquer sur la quantité ou
bien selon la segmentation du marché.''

\subsubsection{Utilité}\label{utilituxe9}

``L'utilité mesure le bien-être liée à la consommation d'un bien.''

\subsubsection{Actualisation}\label{actualisation}

``L'Actualisation est un calcul permettant de transformer une valeur
future en une valeur présente.''

Que vaut aujourd'hui les X euros que j'aurais demain ?

\[
V_a = \frac{V_f}{(1+i)^t}
\]

\[
V_a : Valeur\ Actuelle
\]

\[
V_f : Valeur\ future
\] \[
i : Taux\ sans\ risque\\ 
\]

\[
t : Temps
\]

\subsubsection{Problèmes
macroéconomiques}\label{probluxe8mes-macrouxe9conomiques}

Il existe 4 grands problèmes macroéconomiques :

\begin{itemize}
\item
  Crises et récessions Ralentissement et/ou régression de l'activité
  économique ;
\item
  Inflations Augmentation générale et durable du niveau des prix
  entraînant une perte du pouvoir d'achat de la monnaie ;
\item
  Chômage Inactivité due au manque de travail ;
\item
  Problème de l'équilibre extérieur Quand les importations sont plus
  importantes que les exportations, la balance commerciale est
  déséquilibrée.
\end{itemize}

\subsection{La dissertation en
économie}\label{la-dissertation-en-uxe9conomie}

\emph{``On tient tout d'abord à remercier l'enseignant chercheur (en
Philosophie économique, Théories économiques de la justice,
Redistribution des revenus, Economie sociale, Economie publique),
monsieur \textbf{Jean-Sébastien Gharbi} pour cette rubrique d'aide à la
dissertation.''}

\begin{center}
\includegraphics[width=3.33333in,height=\textheight,keepaspectratio]{images/dissertation.png}
\end{center}

On pense souvent que la dissertation en économie est un exercice
difficile et qui récompense mal le travail. C'est totalement faux. La
dissertation est un exercice dans lequel il est facile d'obtenir la
moyenne, et même (avec un peu d'entraînement) d'obtenir systématiquement
de très bonnes notes. C'est un exercice relativement facile parce que
c'est un exercice en très grande partie formel~: tout est une question
de méthode.

Faire une dissertation, c'est \emph{montrer que vous êtes capable
d'utiliser et de réorganiser vos connaissances pour répondre à une
question de manière argumentée (c'est-à-dire sous la forme d'un
raisonnement)}. Autrement dit~:

\begin{itemize}
\tightlist
\item
  \textbf{\emph{Une dissertation n'est pas une question de cours.}}
\end{itemize}

La première chose à faire, c'est de différencier question de cours et
dissertation (qui sont souvent confondues). Une question de cours
demande simplement de réciter un cours. Si vous ne faites que réciter
votre cours dans un exercice de dissertation, vous aurez une mauvaise
note. Pourquoi~? Parce que \emph{l'exercice de dissertation suppose de
montrer que vous êtes capable d'utiliser et de réorganiser vos
connaissances} (dans un temps limité) -- pas seulement de réciter une
leçon apprise plus ou moins par cœur. Comme la question de cours, la
dissertation suppose donc que vous savez des choses sur le sujet, mais
il est important de comprendre que la dissertation porte tout autant sur
votre aptitude à organiser vos idées que sur vos connaissances.

\begin{itemize}
\tightlist
\item
  \textbf{\emph{Dans une dissertation, la réponse donnée n'est pas
  importante !}}
\end{itemize}

Une dissertation consiste toujours à répondre à une question. Les
étudiants pensent parfois qu'il y a une «~bonne~» réponse à la question
posée -- qu'il s'agirait de trouver. D'ailleurs, cela contribue à l'idée
(fausse) que la dissertation est un exercice aléatoire~: si vous ne
trouvez pas la bonne réponse, vous avez perdu. Cela aussi est faux~:
\emph{il n'y a (en général) pas de «~bonne~» réponse à la question posée
par la dissertation}. L'exercice de dissertation vient de la
philosophie. Pensez-vous sérieusement que l'on puisse demander à un
étudiant (ou à un professeur, d'ailleurs) de régler de façon définitive
un débat philosophique qui a donné lieu à des controverses pendant des
siècles en trois, quatre ou même sept heures~? La réponse évidente est
«~non~».

\begin{itemize}
\tightlist
\item
  \textbf{\emph{Dans une dissertation, le plus important c'est
  l'argumentation !}}
\end{itemize}

Si on ne s'intéresse pas à la réponse donnée. C'est tout simplement,
parce que \emph{ce qui intéresse votre lecteur, c'est la manière dont
vous répondez} : votre aptitude à utiliser vos connaissances de manière
argumentative pour défendre une conclusion. Sur le principe, il serait
donc possible de défendre une conclusion choquante ou même offensante
dans une dissertation, pour la bonne raison qu'on n'évalue pas la
réponse que vous donnez, mais la manière dont vous amenez votre réponse.
Votre réponse, à la limite, on ne s'y intéresse pas. Une fois cela dit,
il est assez évident qu'il est beaucoup plus facile de défendre une
position modérée et consensuelle, qu'une position offensante pour de
nombreuses personnes. C'est la raison pour laquelle, il n'est pas du
tout conseillé de chercher la provocation gratuite dans une
dissertation.

\ul{\textbf{Comment on fait une dissertation ?}}

\begin{center}
\includegraphics[width=1.8125in,height=\textheight,keepaspectratio]{images/question-02.png}
\end{center}

\subsubsection{Analyse du sujet}\label{analyse-du-sujet}

L'analyse du sujet est la première étape de la dissertation et l'une des
plus importantes. Une dissertation se présente sous la forme d'un sujet.
Il faut isoler la ou les deux notions principales du sujet.

Il y a quatre grands types de sujets~: les sujets ne contenant qu'une
seule notion (ex~: «~Les discriminations en France~»), les couples de
notions (ex~: «~Capitalisme et démocratie~»), les citations (ex~:
«~\emph{Le système de production capitaliste est une démocratie
économique dans laquelle chaque sou donne un droit de vote. Les
consommateurs constituent le peuple souverain}~», Ludwig von Mises) ou
une question (ex~: «~La croissance économique s'oppose-t-elle à la
préservation de l'environnement~?~»).

Dans tous les cas, l'objectif est d'arriver à une question (donc les
sujets les plus simples à traiter à ce stade, ce sont les questions~:
ils vous donnent immédiatement le problème à traiter). Mais la première
chose à faire (même quand on a déjà la question), c'est de trouver le
couple de notions impliquées dans le sujet. Souvent, c'est absolument
évident, mais parfois il faut un peu chercher.

Il faut éviter à tout prix de faire un exposé quand on attend de vous
une dissertation. Un hors-sujet, c'est de ne pas traiter le bon sujet.
Si vous répondez de manière factuelle à un sujet de dissertation, vous
faites pire~: vous faites un hors-exercice.

\subsubsection{Recherche des idées}\label{recherche-des-iduxe9es}

Une fois qu'on a identifié un couple de notions, il faut (au brouillon)
essayer de faire la liste des éléments du cours qui relient les deux
notions. Il est important de ne noter que les éléments qui relient les
deux notions (pour ne pas risquer de se perdre dans des éléments qui
concernent seulement une seule des deux notions). Sur chacune des
notions que vous aurez à traiter en dissertation, on a écrit des livres
entiers. Il est impossible de tout dire dessus dans une dissertation. On
se limite donc à ce qui relie les deux notions de notre sujet.
Évidemment, si un élément pertinent vous vient en tête et qu'il ne se
trouve pas dans votre cours, n'hésitez pas à le noter. Si cet élément
peut être utilisé dans votre raisonnement, même comme exemple, ce sera
un plus indiscutable.

Dans un premier temps, on note tout ce qui se présente à l'esprit. Ce
n'est que dans un deuxième temps, quand on a un certain nombre
d'éléments que l'on se pose la question~: «~Est-ce qu'il y a une manière
qui saute aux yeux de relier tous ces éléments en répondant à la
question (si elle a été posée de manière directe) ou pour répondre à une
question comprenant le couple de notions (si la question n'a pas été
formulée dans le sujet)~? ». Si la réponse est positive, on a trouvé la
question qui structurera notre devoir. Si ce n'est pas le cas, il faut
essayer de trouver une question qui relie le plus grand nombre des
éléments que l'on a noté sur son brouillon~-- et donc laisser de côté
les éléments qui ne servent pas. Il arrive souvent qu'une partie des
éléments que l'on note sur son brouillon ne soit pas utilisée dans le
devoir. Bref, si notre sujet est un couple de notions ou une citation,
il faut que l'on arrive à une question. Évidemment, quand notre sujet
est déjà une question, on n'a pas autant de marge de manœuvre, mais en
vérité, si on vous pose une question précise, c'est que vous avez les
éléments pour y répondre dans le cours (donc la différence n'est pas
très importante).

\subsubsection{Mise en évidence d'un
problème}\label{mise-en-uxe9vidence-dun-probluxe8me}

Puisqu'elle ne doit pas être une question de fait, la question qui relie
le plus d'éléments possibles parmi ceux qui associent les deux notions
dans votre cours doit être une question conceptuelle. Pour le dire
autrement, cette question doit être un problème (on parle souvent de
«~problématique~» pour désigner ce problème dans le cadre d'une
dissertation). Qu'est-ce qu'un problème~? C'est une question qui met en
tension deux concepts et qui analyse les différents aspects de leur
relation (conceptuelle).

Souvent les étudiants ont peur de ne pas trouver le «~bon~» problème.
Pourtant, si on suit la méthode de dissertation, il n'y a pas de risque
de se tromper. En effet, on ne doit pas choisir un problème d'abord
(sans savoir si on a de quoi le traiter) et le traiter ensuite. Vous
aurez noté qu'on procède exactement dans le sens inverse~: on voit à
quelle question on peut répondre avec les éléments qu'on a sur son
brouillon et on pose précisément la question à laquelle on sait qu'on
peut répondre.

\subsubsection{Construction du plan}\label{construction-du-plan}

Pour la même raison qu'au-dessus, la construction du plan ne doit pas
être très difficile~: il s'agit de rassembler les différents éléments
qui permettent de répondre à la question (au problème que l'on va poser)
de façon à y apporter une réponse.

On va apporter la réponse que les éléments disponibles nous permettent
d'atteindre. Il y a une seule contrainte~: votre plan doit être
suffisamment détaillé. Comme l'objectif de la dissertation, c'est de
montrer que vous êtes capable d'utiliser et de réorganiser vos
connaissances. Si vous ne faites que deux parties sans sous-parties à
l'intérieur (il n'y aurait donc qu'une seule articulation logique), on
trouvera que vous n'avez pas assez structuré votre devoir. Le découpage
minimal, c'est d'avoir quatre éléments (en général, on fait deux grandes
parties avec deux sous-parties chacune, donc on a trois articulations).

Vos parties et vos sous-parties doivent correspondre à des étapes de
votre raisonnement (on dit souvent qu'il faut une idée par sous-partie),
donc votre plan doit donner la structure du raisonnement grâce auquel
vous allez répondre à la question posée. Le plan (qui est annoncé à la
fin de l'introduction et qui doit être apparent dans le devoir, nous
reviendrons sur ce point un peu plus loin) doit permettre de comprendre
la structure de votre devoir d'un coup d'œil -- simplement en lisant les
titres.

Un élément qui permet de savoir si votre plan est bon, c'est de se
demander si à la fin de la première partie, on est arrivé à un état de
la réflexion différent de celui de la fin de l'introduction.

Qu'est-ce que la première partie a permis de comprendre~? Et est-ce que
la seconde partie apporte quelque chose d'autre~? Si chaque partie
représente une étape dans un raisonnement et que votre devoir complet
est donc un raisonnement, votre plan est forcément bon~: vu que c'est
précisément ce qu'on attend de vous.

Il ne faut jamais faire deux sous-parties dans une partie sous la forme
d'une seule phrase coupée par des points de suspension (ex~: «~A)
L'organisation scientifique du travail a permis la croissance des trente
glorieuses\ldots~», «~B) mais, elle a aussi eu des conséquences
négatives, notamment sur le plan social~»). En effet, cela revient à
pointer du doigt que vous opérez une coupure arbitraire (donc que vous
n'articulez pas de manière assez nette les différentes parties de votre
devoir). Sur le principe, les deux sous-parties sont reliées par les
points de suspension et donc ne forment qu'une partie sans coupure.
Préférez toujours les titres qui se succèdent sans être grammaticalement
liés les uns aux autres. Dans l'exemple ci-dessus, il suffit de faire
deux phrases pour découper les deux idées. Si on ne peut pas couper
grammaticalement les deux titres, c'est la preuve que l'articulation
pose problème.

Dans l'idéal, un plan est équilibré~: chaque partie comprend le même
nombre de sous-parties que l'autre et elles font à peu près la même
longueur. Et si vous faites plus de sous-parties dans une partie que
dans l'autre, il faut que les sous-parties soient un peu plus longues
dans la partie qui contient moins de sous-parties. Remarquez que si vous
faites un plan avec deux parties, deux sous-parties, ce dernier problème
ne se pose pas.

Un point important et souvent totalement négligé par les étudiants~:
même quand c'est tentant, \emph{on ne fait jamais de plan centré sur les
auteurs}. Un plan par auteurs conduit très souvent à suivre l'ordre
chronologique et à présenter les positions des auteurs sans les
confronter réellement les unes aux autres . Ce qui doit structurer le
plan, ce sont les concepts (c'est-à-dire les notions qui nous avaient
permis de construire le problème à résoudre).

En réalité, il n'est pas rare qu'on suive plus ou moins l'ordre
chronologie, mais il est essentiel de se focaliser sur les concepts, et
pas sur les auteurs. Pourquoi~? Parce que l'enchaînement ou l'opposition
de concepts constitue un raisonnement (ce que vous devez faire~!), alors
que l'enchaînement ou l'opposition d'auteurs constitue un exposé (ce que
vous ne devez pas faire~!). En réalité, c'est assez facile à faire il
suffit de s'interdire de mentionner le nom des auteurs dans les titres
de partie ou de sous-partie.

Vu que l'objectif du plan, c'est de répondre à une question qui met les
deux notions du sujet en relation, \emph{les parties (ou les
sous-parties) qui se focalisent sur une seule des deux notions sont à
éviter à tout prix~: elles sont simplement hors-sujet}. Sur un sujet
comme «~capitalisme et démocratie~», le plan «~première partie~:
capitalisme~», «~deuxième partie~: démocratie~» est parmi les pires
possibles.

\textbf{Jusqu'à présent, nous n'avons encore rien écrit sur la copie
elle-même.} Nous n'avons travaillé que sur le brouillon. Nous avons deux
notions clés, un problème et un plan. Ce sont les éléments fondamentaux
du devoir. Il faut à présent passer à la rédaction. Nous allons nous
intéresser d'abord à l'introduction.

\subsubsection{La rédaction}\label{la-ruxe9daction}

\ul{\textbf{(Introduction, Conclusion, Développement)}}

\textbf{L'introduction} est la partie la plus importante de la
dissertation. Elle permet de savoir pourquoi le problème se pose,
comment il se pose et comment il va être résolu. A quoi sert
l'introduction~?

\emph{Le rôle de l'introduction, sa raison d'être, c'est de construire
et d'énoncer le problème (la problématique)} auquel le reste du devoir
va répondre. Il ne suffit donc pas de poser la question (pour cela deux
lignes suffiraient) et de commencer le développement. L'introduction,
comme son nom le dit très bien, va introduire le problème, c'est-à-dire
qu'elle va nous y amener, rapidement, certes, mais en plusieurs étapes
très codifiées.

Une introduction de dissertation comprend obligatoirement (au minimum)
cinq éléments~: une accroche, une définition des termes du sujets, la
construction du problème, l'énoncé du problème et l'annonce du plan.
Comme une introduction de dissertation fait entre 20 lignes et une page
et demie (grand maximum), il faut être efficace.

\begin{itemize}
\tightlist
\item
  \textbf{\emph{L'accroche}}
\end{itemize}

\textbf{\emph{Une introduction de dissertation suit des règles assez
rigides. Elle commence toujours par une accroche}}.

\emph{Une «~accroche~», c'est une phrase ou deux qui vont contenir la ou
les deux notion(s) du sujet}. Son rôle est d'amener par étapes le
lecteur vers la question que vous allez poser. Elle sert donc
d'introduction à l'introduction. Une accroche peut être une citation (il
y en a toujours dans un cours) ou un fait récent (le chômage a-t-il
baissé récemment~? Un candidat à l'élection présidentielle a-t-il dit
qu'il fallait juger sa politique en fonction de son impact sur le niveau
de chômage~?). Si on n'a pas de citation ou de fait relevant de
l'actualité, on peut amener le sujet de façon plus habituelle.

Il est important d'éviter un certain nombre de formules toutes faites et
souvent utilisées comme «~De tous temps\ldots~», «~De tous temps, les
hommes\ldots~» ou encore les affirmations très générales (et que vous ne
justifierez pas) comme~: «~Le chômage est un phénomène économique
important, c'est pourquoi il faut l'étudier~». Le défaut de tous ces
débuts d'accroche, c'est qu'ils peuvent servir pour n'importe quel sujet
et que cela se voit.

On met une accroche parce que cela permet de mentionner les termes du
sujet sans commencer directement par une définition -- ce qui constitue
l'étape suivante.

\begin{itemize}
\tightlist
\item
  \textbf{Définition des termes du sujet}
\end{itemize}

L'accroche a introduit les notions, mais sans les définir -- comme si
tout le monde savait précisément de quoi il s'agit (ce qui n'est pas si
surprenant, on ne passe pas son temps à définir tous les mots qu'on
utilise). Mais, pour utiliser les deux notions du sujet de façon un peu
plus précise, il faut les définir. Les définitions que l'on va donner
dans une introduction n'ont pas pour objectif de définir les notions de
manière exhaustive ou dans l'absolu. Elles doivent permettre de
comprendre le lien (ou l'opposition) entre les deux notions et orienter
l'introduction de façon à ce que l'on puisse construire le problème --
avant de l'énoncer (autrement dit, elles doivent ouvrir la voie aux deux
étapes suivantes de l'introduction).

Du coup, les définitions que l'on va donner vont dépendre du problème
que l'on souhaite atteindre.

\begin{itemize}
\tightlist
\item
  \textbf{Construction du problème}
\end{itemize}

Comme on sait à quelle question on doit arriver (que cette question nous
ait été donnée par le sujet ou que ce soit la question à laquelle on est
le mieux armé pour apporter une réponse), il ne va pas être difficile de
passer des définitions au problème. Cela suppose juste de montrer
qu'avec les définitions que l'on vient de donner, il y a une question se
pose avec force.

Encore une fois, cela peut sembler très artificiel (et ça l'est).
Toutefois, l'intérêt de cet aspect artificiel, c'est qu'il nous garantit
que l'on ne va pas se perdre en chemin. Quand on fait une dissertation,
on ne cherche pas son chemin~: on sait où on va et on ne fait
qu'expliquer pourquoi on y va. Le sujet que l'on construit ne tombe pas
du ciel, il vient de notre cours. Les définitions ne tombent pas du
ciel, elles donnent les éléments qui vont nous permettre de poser la
question à laquelle on sait déjà comment on va répondre. Bref, l'étape
de construction du problème est importante parce qu'elle montre que vous
avez des aptitudes pour vous faire comprendre à l'écrit (et il ne faut
surtout pas la négliger), mais elle n'est pas une étape difficile ou
magique.

Vous pourriez être surpris que l'on construise le sujet, alors qu'il
nous est parfois donné sous forme de question (dans les autres cas, on
comprend mieux pourquoi il est nécessaire de construire le problème). En
fait, c'est une manière de montrer que vous êtes capable de vous
approprier le sujet. Vous ne traitez pas le sujet parce qu'on vous l'a
donné (même si vous c'est une des raisons pour lesquelles vous faites
une dissertation), mais parce que vous comprenez pourquoi la question se
pose. Et comment mieux montrer qu'on comprend un problème qu'en montrant
en quoi il est problématique ? Autrement dit, même quand votre sujet a
la forme d'une question, vous devez passer par l'étape de construction
du sujet dans l'introduction.

\begin{itemize}
\tightlist
\item
  \textbf{Enoncé du problème}
\end{itemize}

L'énoncé du problème doit prendre la forme d'une question. Il est le
point final de l'étape juste précédente. Une fois qu'on a les éléments
qui permettent de comprendre que le problème se pose, il faut
explicitement exprimer le problème lui-même. \emph{On exprime toujours
le problème sous la forme d'une question (parce que c'est une manière de
montrer qu'il appelle une réponse) et d'une question unique}. Poser
deux, trois ou quatre questions ce serait soit redire plusieurs fois la
même chose (et si votre première question est claire, c'est inutile),
soit poser (volontairement ou pas) plusieurs questions différentes. Or,
vous ne pourrez pas répondre convenablement et dans les règles de la
dissertation à plusieurs questions en un seul devoir. Vous devrez donc
choisir entre ne pas traiter certaines des questions que vous avez
explicitement posées (et dans ce cas pourquoi les poser explicitement)
ou essayer de les traiter toutes (ce qui vous conduira à un devoir dont
la ligne directrice sera au mieux difficile à suivre, au pire
inexistante). Si on se rappelle du côté formel et rhétorique d'une
dissertation, on comprend qu'il ne faut poser qu'une seule question~:
celle à laquelle vous apportez une réponse.

Lorsque le sujet est une question, faut-il répéter mot pour mot le sujet
comme énoncé du problème~? Il y a deux écoles~: la première dit qu'il
faut reformuler la question pour montrer que vous la comprenez. Ainsi un
sujet comme «~Les dépenses publiques permettent-elles de réduire le
chômage~?~», on pourrait proposer une problématique comme «~les dépenses
publiques sont-elles efficaces à court et à long terme pour lutter
contre le chômage~?~».

Si vous faites correctement votre travail de définition des termes et de
construction du sujet (dans les deux étapes précédentes), je pense
qu'aucun correcteur ne vous reprochera de reprendre le sujet mot pour
mot dans votre énoncé du problème. Ce qui pose problème pour les
partisans de la première façon de faire, c'est quand on peut se demander
si l'étudiant comprend que la question qu'il pose est un problème
conceptuel, c'est-à-dire qui vient d'une tension entre deux notions.
Dans une introduction qui remplit correctement son rôle de construction
du problème, le fait de répéter le sujet mot pour mot n'est pas un
souci.

\begin{itemize}
\tightlist
\item
  \textbf{Annonce du plan}
\end{itemize}

Une introduction doit toujours se terminer par une annonce du plan (ce
n'est pas une option, c'est une obligation). L'annonce de plan dit à
votre lecteur comment vous allez répondre au problème que vous venez de
poser. Dans une dissertation, on ne joue pas sur le suspens. On ne
cherche pas à surprendre son correcteur. Il faut donc annoncer le plan
de manière à ce qu'il comprenne que vous allez répondre au problème posé
par un raisonnement et qu'il comprenne aussi quels vont être les
principales étapes de votre raisonnement (c'est-à-dire de votre devoir).

Vous allez donc annoncer vos (deux ou trois) grandes parties. Il est
conseillé fortement d'utiliser les formules (un peu lourdes en termes de
style, mais très claires) «~dans une première partie, nous montrerons
que\ldots~», puis «~dans une deuxième partie, nous verrons que \ldots~».
Quand vous ne le faites pas, il arrive trop souvent que votre lecteur ne
sache pas si vous allez faire formellement deux ou trois parties -- pour
peu que vous utilisiez des mots comme «~et~», «~puis~» ou «~ensuite~»,
qui peuvent aussi bien marquer des étapes à l'intérieur d'une grande
partie que le passage d'une partie à une autre.

\textbf{La conclusion}

\textbf{Vous pourriez être surpris de voir la conclusion arriver aussi
tôt dans le devoir.} La raison, c'est qu'il est inconcevable de ne pas
répondre à la question posée en introduction -- si vous ne répondez pas
le devoir n'aura, littéralement, servi à rien. Or, il est évident qu'en
partiel, on est souvent pris par le temps. On rédige donc la conclusion
juste après avoir rédigé l'introduction au brouillon (on la rédige aussi
au brouillon, d'ailleurs). Comme ça si on est pris par le temps, on
pourra recopier la conclusion déjà prête avant de rendre le devoir. S'il
faut couper quelque chose en raison du temps limité de l'épreuve, il
vaut mieux couper un bout du développement que rendre une dissertation
sans conclusion.

La première phrase de votre conclusion doit apporter la réponse à la
question que vous avez posée en introduction. Elle doit le faire de
façon absolument claire et donc il est conseillé de reprendre exactement
la question en la tournant en une phrase affirmative ou en une phrase
négative selon votre réponse. \emph{Le rôle de la conclusion, c'est de
répondre à la question}. Il ne faut pas qu'on relise la conclusion en se
demandant quelle était la réponse -- et même en se demandant si une
réponse a été donnée. Cela ne vous empêche pas de donner une réponse
nuancée, mais il faut une réponse claire.

Une conclusion de dissertation ne résume pas le devoir (on vient de le
lire, c'est tout à fait inutile). Une conclusion n'introduit jamais un
élément qui n'a pas été abordé dans le devoir, mais qui aurait pu y être
discuté. Si jamais votre correcteur n'a pas vu que vous avez oublié de
parler de quelque chose d'important, vous n'allez tout de même pas lui
dire qu'il manque quelque chose dans votre devoir (chacun son boulot).
La dissertation est un exercice de rhétorique, votre objectif, c'est de
convaincre votre lecteur~: ce n'est pas à vous de dire qu'il manque
quelque chose, même si vous le savez.

On conseille parfois de finir sa dissertation sur une ouverture. Une
ouverture est un nouveau problème qui se pose une fois que vous avez
répondu au problème de votre devoir. Cela revient à suggérer une autre
dissertation possible une fois qu'on considère votre réponse comme
acceptée. Trop souvent, les étudiants finissent leurs devoirs de manière
particulièrement maladroite parce qu'ils ne comprennent pas ce qu'est
une ouverture. Mon conseil est d'éviter de faire une ouverture, au moins
au début~: ce n'est pas une obligation et cela peut donner une très
mauvaise impression finale.

\textbf{La rédaction du devoir}

Une fois tout cela fait, on prend sa copie (totalement vierge à ce
moment) et on commence à écrire dessus~: on recopie l'introduction, on
rédige le développement directement sur la copie (on ne rédige jamais
son développement sur le brouillon, cela prend beaucoup trop de temps à
recopier). \emph{Le développement du devoir doit contenir des titres
apparents pour les parties et les sous-parties}. Cela signifie que le
titre de votre grande partie est marqué dans votre copie (précédé d'un
«~I)~») et qu'il est isolé du texte et souligné. Bref, on doit pouvoir
voir apparaître d'un coup d'œil votre plan en survolant votre copie du
regard.

Comme dit juste au-dessus, si on manque de temps, on coupe une partie du
développement et on recopie la conclusion qui se trouve sur le
brouillon. Attention~: si vous ne rédigez pas tout le développement,
mettez tout de même le plan apparent pour les parties et sous-parties
non développées. C'est précisément parce qu'on a une idée de ce que vous
auriez écrit qu'il est possible (en cas de gros manque de temps) de ne
pas rédiger tout le développement. Si vous ne détaillez pas votre plan,
c'est la trame de votre raisonnement qui manque et c'est beaucoup plus
ennuyeux. Si vous ne pouvez pas rédiger tout le développement, je vous
conseille de mettre des éléments que vous auriez utilisé sous forme de
liste de tirets (en plus des titres apparents qui sont obligatoires).




\end{document}
